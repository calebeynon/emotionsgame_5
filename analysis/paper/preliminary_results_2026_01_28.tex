\documentclass[11pt,oneside]{article}
\usepackage[margin=1in]{geometry}

% Standard Packages
\usepackage{amssymb}
\usepackage{amsmath}
\usepackage{graphicx}
\usepackage[usenames,dvipsnames,svgnames,table]{xcolor}
\usepackage{colortbl}
\usepackage{changepage}
\usepackage{soul}
\usepackage{pdflscape}
\usepackage{float}
\usepackage[font=footnotesize,labelfont=bf]{caption}
\usepackage{framed}
\usepackage{enumitem}
\usepackage[utf8]{inputenc}
\usepackage{graphics}
\usepackage{titlesec}
\usepackage{booktabs}
\usepackage{setspace}
\usepackage{changepage}
\usepackage{multicol}
\usepackage[authoryear]{natbib}
\setlength{\bibsep}{1mm}
\usepackage{multirow}
\usepackage{tocloft}
\usepackage{dcolumn}
\usepackage{outlines}
\usepackage{subcaption}
\captionsetup[subfigure]{labelformat=empty}
\usepackage{tabularray}
\usepackage{bm}
\usepackage{placeins}
\usepackage[bookmarks]{hyperref}
\hypersetup{colorlinks = true, allcolors = {black}}
\usepackage{bookmark}

%%%%%%%%% NN styling %%%%%%%%%
\usepackage{NNstyle}

\linespread{1.3} % 1.5 spacing = 1.3, double spacing = 1.6

\begin{document}

	%%% TITLE/AUTHORS/DATE %%%
	\begin{titlepage}
		\begin{singlespace}
			\begin{centering}
				%%% TITLE/AUTHORS %%%
				\title{\vspace{-1.5cm}Your New Paper Title Goes Here\\[5mm]}
				\author{
					Author One\footnote{Affiliation, \href{mailto:email1@example.com}{email1@example.com}.}
					\and
                    Author Two\footnote{Affiliation, \href{mailto:email2@example.com}{email2@example.com}.}
				}
				\date{}
				\maketitle
			\end{centering}
		\end{singlespace}

		\thispagestyle{empty}
		\vspace{2mm}
		\begin{centr}
			{-- \albf{PRELIMINARY DRAFT} -- \\ \al{PLEASE DO NOT CITE OR DISTRIBUTE}}\\[-1mm]{\small Version: 2026-0128}
		\end{centr}
		\vspace{3mm}
		%%% ABSTRACT %%%
		\begin{abstract}
			\begin{singlespace}
					\vspace{2mm}
					\noindent Replace this with your paper's abstract.
			\end{singlespace}
			%%% KEYWORDS %%%
			\vspace{10mm}
			\noindent \textit{Keywords:} Public goods, Communication, Promises, Sentiment, Cooperation

			%%% JEL CLASSIFICATIONS %%%
			% \noindent \textit{JEL Classifications:} C72 $\cdot$ C92 $\cdot$ D71
		\end{abstract}
	\end{titlepage}

	\clearpage
	\pagestyle{plain}
	\pagenumbering{arabic}

	%%%%%%%%%%%%%%%%%%%%%%%%%%%%%
	% PRELIMINARY RESULTS
	%%%%%%%%%%%%%%%%%%%%%%%%%%%%%

	\section{Preliminary Results}

	This document summarizes preliminary results from a public goods experiment with chat communication. The experiment features 16 participants per session across 10 sessions (5 per treatment), organized into groups of 4 with a 25-point endowment and a 0.4 MPCR. Participants play 5 sequential supergames of varying length (3--7 rounds), with strategic regrouping between supergames and pre-decision chat in each round.


	%%% 1. DESCRIPTIVE STATISTICS %%%
	\subsection{Contribution Summary Statistics}

	Table~\ref{tab:contributions_aggregate} reports mean contributions by treatment and supergame. Treatment~2 participants contribute more on average across all five supergames. Mean contributions in Treatment~1 range from 17.44 (supergame~1) to 20.46 (supergame~5), while Treatment~2 contributions range from 19.18 to 22.02 over the same span. Median contributions reach the endowment ceiling of 25 in most treatment-supergame cells.

	\begin{table}[H]
		\centering
		\caption{Contribution Summary Statistics by Treatment and Supergame}
		\label{tab:contributions_aggregate}
		% latex table generated in R 4.5.0 by xtable 1.8-4 package
% Sun Jan 25 15:28:58 2026
\begin{tabular}{clrrrrrr}
  \toprule
Treatment & Segment & N & Mean & SD & Min & Max & Median \\ 
  \midrule
  1 & supergame1 & 240 & 17.44 & 8.35 & 0.00 & 25.00 & 20.00 \\ 
    1 & supergame2 & 320 & 20.12 & 8.20 & 0.00 & 25.00 & 25.00 \\ 
    1 & supergame3 & 240 & 19.25 & 9.13 & 0.00 & 25.00 & 25.00 \\ 
    1 & supergame4 & 560 & 18.98 & 10.10 & 0.00 & 25.00 & 25.00 \\ 
    1 & supergame5 & 400 & 20.46 & 9.06 & 0.00 & 25.00 & 25.00 \\ 
    2 & supergame1 & 240 & 19.18 & 8.53 & 0.00 & 25.00 & 25.00 \\ 
    2 & supergame2 & 320 & 21.01 & 7.84 & 0.00 & 25.00 & 25.00 \\ 
    2 & supergame3 & 240 & 20.25 & 8.69 & 0.00 & 25.00 & 25.00 \\ 
    2 & supergame4 & 560 & 20.83 & 8.68 & 0.00 & 25.00 & 25.00 \\ 
    2 & supergame5 & 400 & 22.02 & 7.36 & 0.00 & 25.00 & 25.00 \\ 
   \bottomrule
\end{tabular}

	\end{table}


	%%% 2. BEHAVIORAL CLASSIFICATIONS %%%
	\subsection{Promise-Making and Behavioral Classifications}

	Chat messages are classified as promises using GPT-5-mini with context-aware prompting. From promise classifications, we derive two additional behavioral measures:
	\begin{itemize}[nosep]
		\item \textbf{Liar.} A player who made a promise in chat but subsequently contributed below a threshold. Under the \textit{strict} definition, a liar is a promise-maker who contributed $< 20$ (out of 25); under the \textit{lenient} definition, a liar contributed $< 5$. The lenient definition captures only extreme defection.
		\item \textbf{Sucker.} A player who contributed the maximum (25 points) in a round where a groupmate lied. Under the \textit{strict} definition, a groupmate lied if they promised but contributed $< 20$; under the \textit{lenient} definition, the groupmate contributed $< 5$. Because the strict liar threshold is easier to meet, the strict sucker definition classifies \textit{more} players as suckers.
	\end{itemize}

	Table~\ref{tab:behavior_aggregate} summarizes these classifications aggregated across sessions within each treatment. Promise counts are broadly similar across treatments (519 in Treatment~1 vs.\ 412 in Treatment~2 across all supergames). Lying and sucker incidence is concentrated in the longer supergames (supergame~4 in particular), suggesting that cooperation breakdowns are more likely in extended interactions.

	\begin{table}[H]
		\centering
		\caption{Behavioral Classifications by Treatment and Supergame}
		\label{tab:behavior_aggregate}
		% latex table generated in R 4.5.0 by xtable 1.8-4 package
% Sun Feb  1 15:23:48 2026
\begin{tabular}{clrrrrrr}
  \toprule
Treatment & Segment & N & Promises & Liars 20 & Liars 5 & Suckers 20 & Suckers 5 \\ 
  \midrule
  1 & supergame1 & 240 &  75 &  14 &   1 &   5 &   1 \\ 
    1 & supergame2 & 320 &  95 &  12 &   3 &  17 &   9 \\ 
    1 & supergame3 & 240 &  89 &   2 &   1 &   6 &   3 \\ 
    1 & supergame4 & 560 & 123 &  19 &  18 &  33 &  33 \\ 
    1 & supergame5 & 400 & 137 &  10 &   7 &  25 &  19 \\ 
    2 & supergame1 & 240 &  77 &   4 &   3 &   9 &   9 \\ 
    2 & supergame2 & 320 &  79 &   6 &   4 &   8 &   4 \\ 
    2 & supergame3 & 240 &  63 &   4 &   3 &   8 &   5 \\ 
    2 & supergame4 & 560 & 104 &  35 &  30 &  72 &  58 \\ 
    2 & supergame5 & 400 &  89 &  17 &   5 &  22 &  13 \\ 
   \bottomrule
\end{tabular}

	\end{table}


	%%% 3. CONTRIBUTION FIGURES %%%
	\subsection{Contribution Dynamics}

	\begin{figure}[H]
		\centering
		\includegraphics[width=0.85\textwidth]{plots/mean_contribution_by_round.png}
		\caption{Mean contribution by round within each supergame. Vertical breaks separate supergames.}
		\label{fig:mean_by_round}
	\end{figure}

	\begin{figure}[H]
		\centering
		\includegraphics[width=0.85\textwidth]{plots/mean_contribution_by_segment.png}
		\caption{Mean contribution by supergame, split by treatment.}
		\label{fig:mean_by_segment}
	\end{figure}

	\begin{figure}[H]
		\centering
		\includegraphics[width=0.85\textwidth]{plots/contribution_cdf_by_treatment.png}
		\caption{Empirical CDF of contributions by treatment. The Treatment~2 distribution first-order stochastically dominates Treatment~1.}
		\label{fig:cdf}
	\end{figure}

	\FloatBarrier


	%%% 4. REGRESSION: PROMISE AND SUCKER EFFECTS %%%
	\subsection{Contribution Regression: Promise and Sucker Effects}

	Table~\ref{tab:contribution_regression} reports fixed-effects regressions of individual contributions on promise-making, sucker status, and treatment assignment. The model includes round and supergame (segment) fixed effects, with standard errors clustered at the session-segment-group level.

	\textbf{Key findings:}
	\begin{itemize}[nosep]
		\item \textit{Promise-making} is not significantly associated with higher contributions ($\hat{\beta} \approx 0.53$, $p > 0.10$). This likely reflects a ceiling effect: the contribution distribution is heavily right-skewed, with median contributions at the 25-point endowment in most treatment-supergame cells (Table~\ref{tab:contributions_aggregate}). Both promise-makers and non-promise-makers tend to contribute near the maximum, leaving insufficient variation for the promise indicator to explain. The promise variable thus captures a near-universal norm of high contribution rather than a behavioral distinction.
		\item \textit{Sucker status} has a large, negative, and highly significant effect. Under the strict definition (groupmate contributed $< 20$ after promising), being a sucker reduces contributions by 5.99 points ($p < 0.01$). Under the lenient definition (groupmate contributed $< 5$), the effect is even larger at $-8.02$ points ($p < 0.01$).
		\item \textit{Treatment} is positive and significant ($\hat{\beta} \approx 1.62$, $p < 0.01$), confirming that Treatment~2 participants contribute more than Treatment~1 participants.
	\end{itemize}

	\begin{table}[H]
		\centering
		\caption{Contribution Regression: Promise and Sucker Effects}
		\label{tab:contribution_regression}
		
\begingroup
\centering
\begin{tabular}{lcc}
   \tabularnewline \midrule \midrule
   Dependent Variable: & \multicolumn{2}{c}{contribution}\\
                   & Sucker (<20)   & Sucker (<5) \\   
   Model:          & (1)            & (2)\\  
   \midrule
   \emph{Variables}\\
   Made Promise    & 0.5288         & 0.5230\\   
                   & (0.3320)       & (0.3286)\\   
   Is Sucker (<20) & -5.988$^{***}$ &   \\   
                   & (1.284)        &   \\   
   Treatment       & 1.623$^{***}$  & 1.620$^{***}$\\   
                   & (0.5075)       & (0.4834)\\   
   Is Sucker (<5)  &                & -8.022$^{***}$\\   
                   &                & (1.391)\\   
   \midrule
   \emph{Fixed-effects}\\
   round           & Yes            & Yes\\  
   segment         & Yes            & Yes\\  
   \midrule
   \emph{Fit statistics}\\
   Observations    & 3,520          & 3,520\\  
   R$^2$           & 0.28594        & 0.29499\\  
   \midrule \midrule
   \multicolumn{3}{l}{\emph{Clustered (session-segment-group) standard-errors in parentheses}}\\
   \multicolumn{3}{l}{\emph{Signif. Codes: ***: 0.01, **: 0.05, *: 0.1}}\\
\end{tabular}
\par\endgroup



	\end{table}

	\FloatBarrier


	%%% 5. REGRESSION: SENTIMENT AND LIAR INTERACTION %%%
	\subsection{Sentiment--Contribution Regression with Liar Interaction}

	Table~\ref{tab:sentiment_liar} tests whether chat sentiment predicts contributions and whether this relationship is attenuated for liars. The model interacts mean VADER compound sentiment (computed from pre-decision chat messages) with a liar indicator, including round and segment fixed effects with session-segment-group clustered standard errors.

	The sample is restricted to person-rounds with non-missing chat sentiment ($N = 2{,}298$), which excludes Round~1 of each supergame (where no prior chat exists to influence contributions).

	\textbf{Key findings:}
	\begin{itemize}[nosep]
		\item \textit{Sentiment} positively predicts contributions: a one-unit increase in compound sentiment is associated with roughly 2.1--2.2 additional contribution points ($p < 0.01$). Players who express more positive sentiment in chat contribute more.
		\item \textit{Lying} has a large negative main effect. Strict liars (promised but contributed $< 20$) contribute 17.16 points less ($p < 0.01$); lenient liars (contributed $< 5$) contribute 22.49 points less ($p < 0.01$).
		\item \textit{Interaction (Sentiment $\times$ Lied):} Under the strict definition, the interaction is small and insignificant ($-0.23$, $p > 0.10$), suggesting sentiment remains equally predictive for liars and non-liars. Under the lenient definition, the interaction is negative and significant ($-2.65$, $p < 0.05$), indicating that for extreme liars, positive sentiment does \textit{not} translate into higher contributions---consistent with cheap talk.
		\item \textit{Treatment} is not significant in either specification once sentiment and liar status are controlled, suggesting the treatment effect operates partly through the sentiment and honesty channels.
	\end{itemize}

	\begin{table}[H]
		\centering
		\caption{Sentiment--Contribution Regression with Liar Interaction}
		\label{tab:sentiment_liar}
		
\begingroup
\centering
\begin{tabular}{lcc}
   \tabularnewline \midrule \midrule
   Dependent Variable: & \multicolumn{2}{c}{contribution}\\
                    & Liar (<20)     & Liar (<5) \\   
   Model:           & (1)            & (2)\\  
   \midrule
   \emph{Variables}\\
   Sentiment        & 2.170$^{***}$  & 2.114$^{***}$\\   
                    & (0.5885)       & (0.5894)\\   
   Lied (<20)       & -17.16$^{***}$ &   \\   
                    & (1.054)        &   \\   
   Treatment        & -0.1034        & 0.0569\\   
                    & (0.4883)       & (0.4965)\\   
   Sentiment x Lied & -0.2258        & -2.648$^{**}$\\   
                    & (3.496)        & (1.112)\\   
   Lied (<5)        &                & -22.49$^{***}$\\   
                    &                & (0.5412)\\   
   \midrule
   \emph{Fixed-effects}\\
   round            & Yes            & Yes\\  
   segment          & Yes            & Yes\\  
   \midrule
   \emph{Fit statistics}\\
   Observations     & 2,298          & 2,298\\  
   R$^2$            & 0.31743        & 0.30210\\  
   \midrule \midrule
   \multicolumn{3}{l}{\emph{Clustered (cluster\_id) standard-errors in parentheses}}\\
   \multicolumn{3}{l}{\emph{Signif. Codes: ***: 0.01, **: 0.05, *: 0.1}}\\
\end{tabular}
\par\endgroup



	\end{table}


    \newpage

	%%%%%%%%%%%%%%%%%%%%%%%%%%%%%%%%
	% References
	%%%%%%%%%%%%%%%%%%%%%%%%%%%%%%%%

	\singlespacing
	\nocite{*}

	{\small
		\bibliography{NNbib} % Bibliography file
		\bibliographystyle{apalike}
	}

	\newpage

	\FloatBarrier

	\appendix

    \section{Appendix: Detailed Summary Statistics}

	Table~\ref{tab:contributions_detail} reports contribution statistics for each session-supergame cell. Table~\ref{tab:behavior_detail} provides the corresponding behavioral classification counts.

	\begin{table}[H]
		\centering
		\caption{Contribution Summary Statistics by Session and Supergame}
		\label{tab:contributions_detail}
		\small
		% latex table generated in R 4.5.0 by xtable 1.8-4 package
% Sun Jan 25 15:26:31 2026
\begin{tabular}{cllrrrrrr}
  \toprule
Treatment & Session & Segment & N & Mean & SD & Min & Max & Median \\ 
  \midrule
  1 & 6sdkxl2q & supergame1 &  48 & 17.35 & 9.34 & 0.00 & 25.00 & 25.00 \\ 
    1 & 6sdkxl2q & supergame2 &  64 & 18.77 & 8.75 & 0.00 & 25.00 & 25.00 \\ 
    1 & 6sdkxl2q & supergame3 &  48 & 18.58 & 9.91 & 0.00 & 25.00 & 25.00 \\ 
    1 & 6sdkxl2q & supergame4 & 112 & 17.81 & 10.95 & 0.00 & 25.00 & 25.00 \\ 
    1 & 6sdkxl2q & supergame5 &  80 & 20.85 & 9.11 & 0.00 & 25.00 & 25.00 \\ 
    1 & iiu3xixz & supergame1 &  48 & 14.29 & 8.26 & 0.00 & 25.00 & 13.00 \\ 
    1 & iiu3xixz & supergame2 &  64 & 19.31 & 8.53 & 0.00 & 25.00 & 25.00 \\ 
    1 & iiu3xixz & supergame3 &  48 & 18.94 & 9.08 & 0.00 & 25.00 & 25.00 \\ 
    1 & iiu3xixz & supergame4 & 112 & 18.41 & 10.60 & 0.00 & 25.00 & 25.00 \\ 
    1 & iiu3xixz & supergame5 &  80 & 21.98 & 7.64 & 0.00 & 25.00 & 25.00 \\ 
    1 & r5dj4yfl & supergame1 &  48 & 18.90 & 7.18 & 0.00 & 25.00 & 22.50 \\ 
    1 & r5dj4yfl & supergame2 &  64 & 21.73 & 7.22 & 0.00 & 25.00 & 25.00 \\ 
    1 & r5dj4yfl & supergame3 &  48 & 18.88 & 9.50 & 0.00 & 25.00 & 25.00 \\ 
    1 & r5dj4yfl & supergame4 & 112 & 20.03 & 9.47 & 0.00 & 25.00 & 25.00 \\ 
    1 & r5dj4yfl & supergame5 &  80 & 20.48 & 9.25 & 0.00 & 25.00 & 25.00 \\ 
    1 & sa7mprty & supergame1 &  48 & 17.00 & 7.91 & 0.00 & 25.00 & 20.00 \\ 
    1 & sa7mprty & supergame2 &  64 & 21.56 & 6.35 & 5.00 & 25.00 & 25.00 \\ 
    1 & sa7mprty & supergame3 &  48 & 19.92 & 8.08 & 0.00 & 25.00 & 25.00 \\ 
    1 & sa7mprty & supergame4 & 112 & 20.46 & 8.67 & 0.00 & 25.00 & 25.00 \\ 
    1 & sa7mprty & supergame5 &  80 & 18.80 & 10.15 & 0.00 & 25.00 & 25.00 \\ 
    1 & umbzdj98 & supergame1 &  48 & 19.67 & 8.22 & 0.00 & 25.00 & 25.00 \\ 
    1 & umbzdj98 & supergame2 &  64 & 19.22 & 9.52 & 0.00 & 25.00 & 25.00 \\ 
    1 & umbzdj98 & supergame3 &  48 & 19.94 & 9.30 & 0.00 & 25.00 & 25.00 \\ 
    1 & umbzdj98 & supergame4 & 112 & 18.21 & 10.52 & 0.00 & 25.00 & 25.00 \\ 
    1 & umbzdj98 & supergame5 &  80 & 20.21 & 8.93 & 0.00 & 25.00 & 25.00 \\ 
    2 & 6ucza025 & supergame1 &  48 & 22.54 & 6.00 & 2.00 & 25.00 & 25.00 \\ 
    2 & 6ucza025 & supergame2 &  64 & 23.06 & 5.53 & 0.00 & 25.00 & 25.00 \\ 
    2 & 6ucza025 & supergame3 &  48 & 23.83 & 3.06 & 9.00 & 25.00 & 25.00 \\ 
    2 & 6ucza025 & supergame4 & 112 & 23.68 & 4.38 & 0.00 & 25.00 & 25.00 \\ 
    2 & 6ucza025 & supergame5 &  80 & 24.62 & 1.55 & 15.00 & 25.00 & 25.00 \\ 
    2 & 6uv359rf & supergame1 &  48 & 16.98 & 8.76 & 0.00 & 25.00 & 17.50 \\ 
    2 & 6uv359rf & supergame2 &  64 & 20.42 & 8.11 & 0.00 & 25.00 & 25.00 \\ 
    2 & 6uv359rf & supergame3 &  48 & 20.40 & 8.82 & 0.00 & 25.00 & 25.00 \\ 
    2 & 6uv359rf & supergame4 & 112 & 22.05 & 7.76 & 0.00 & 25.00 & 25.00 \\ 
    2 & 6uv359rf & supergame5 &  80 & 20.82 & 9.01 & 0.00 & 25.00 & 25.00 \\ 
    2 & irrzlgk2 & supergame1 &  48 & 18.81 & 9.51 & 0.00 & 25.00 & 25.00 \\ 
    2 & irrzlgk2 & supergame2 &  64 & 18.62 & 10.16 & 0.00 & 25.00 & 25.00 \\ 
    2 & irrzlgk2 & supergame3 &  48 & 16.31 & 11.29 & 0.00 & 25.00 & 25.00 \\ 
    2 & irrzlgk2 & supergame4 & 112 & 14.72 & 11.65 & 0.00 & 25.00 & 25.00 \\ 
    2 & irrzlgk2 & supergame5 &  80 & 20.44 & 8.79 & 0.00 & 25.00 & 25.00 \\ 
    2 & j3ki5tli & supergame1 &  48 & 16.83 & 9.12 & 0.00 & 25.00 & 20.00 \\ 
    2 & j3ki5tli & supergame2 &  64 & 20.34 & 7.52 & 0.00 & 25.00 & 25.00 \\ 
    2 & j3ki5tli & supergame3 &  48 & 19.06 & 8.68 & 0.00 & 25.00 & 25.00 \\ 
    2 & j3ki5tli & supergame4 & 112 & 22.58 & 6.55 & 0.00 & 25.00 & 25.00 \\ 
    2 & j3ki5tli & supergame5 &  80 & 22.79 & 5.68 & 0.00 & 25.00 & 25.00 \\ 
    2 & sylq2syi & supergame1 &  48 & 20.73 & 7.66 & 0.00 & 25.00 & 25.00 \\ 
    2 & sylq2syi & supergame2 &  64 & 22.61 & 6.43 & 0.00 & 25.00 & 25.00 \\ 
    2 & sylq2syi & supergame3 &  48 & 21.65 & 7.87 & 0.00 & 25.00 & 25.00 \\ 
    2 & sylq2syi & supergame4 & 112 & 21.13 & 8.42 & 0.00 & 25.00 & 25.00 \\ 
    2 & sylq2syi & supergame5 &  80 & 21.41 & 8.27 & 0.00 & 25.00 & 25.00 \\ 
   \bottomrule
\end{tabular}

	\end{table}

	\begin{table}[H]
		\centering
		\caption{Behavioral Classifications by Session and Supergame}
		\label{tab:behavior_detail}
		\small
		% latex table generated in R 4.5.0 by xtable 1.8-4 package
% Sun Jan 25 15:12:11 2026
\begin{tabular}{lclrrrrrr}
  \toprule
Session & Treatment & Segment & N & Promises & Liars Strict & Liars Lenient & Suckers Strict & Suckers Lenient \\ 
  \midrule
6sdkxl2q &   1 & supergame1 &  48 &  19 &   3 &   0 &   0 &   0 \\ 
  6sdkxl2q &   1 & supergame2 &  64 &  21 &   5 &   0 &   6 &   0 \\ 
  6sdkxl2q &   1 & supergame3 &  48 &  17 &   0 &   0 &   0 &   0 \\ 
  6sdkxl2q &   1 & supergame4 & 112 &  32 &   2 &   2 &   1 &   1 \\ 
  6sdkxl2q &   1 & supergame5 &  80 &  30 &   0 &   0 &   0 &   0 \\ 
  iiu3xixz &   1 & supergame1 &  48 &  10 &   2 &   1 &   1 &   1 \\ 
  iiu3xixz &   1 & supergame2 &  64 &  19 &   4 &   1 &   3 &   3 \\ 
  iiu3xixz &   1 & supergame3 &  48 &  20 &   1 &   1 &   3 &   3 \\ 
  iiu3xixz &   1 & supergame4 & 112 &  26 &   4 &   4 &  12 &  12 \\ 
  iiu3xixz &   1 & supergame5 &  80 &  30 &   0 &   0 &   0 &   0 \\ 
  r5dj4yfl &   1 & supergame1 &  48 &  24 &   4 &   0 &   1 &   0 \\ 
  r5dj4yfl &   1 & supergame2 &  64 &  21 &   0 &   0 &   0 &   0 \\ 
  r5dj4yfl &   1 & supergame3 &  48 &  20 &   1 &   0 &   3 &   0 \\ 
  r5dj4yfl &   1 & supergame4 & 112 &  24 &   2 &   2 &   6 &   6 \\ 
  r5dj4yfl &   1 & supergame5 &  80 &  32 &   3 &   0 &   6 &   0 \\ 
  sa7mprty &   1 & supergame1 &  48 &  11 &   4 &   0 &   0 &   0 \\ 
  sa7mprty &   1 & supergame2 &  64 &  25 &   0 &   0 &   0 &   0 \\ 
  sa7mprty &   1 & supergame3 &  48 &  18 &   0 &   0 &   0 &   0 \\ 
  sa7mprty &   1 & supergame4 & 112 &  24 &   5 &   5 &   8 &   8 \\ 
  sa7mprty &   1 & supergame5 &  80 &  22 &   4 &   4 &  10 &  10 \\ 
  umbzdj98 &   1 & supergame1 &  48 &  11 &   1 &   0 &   3 &   0 \\ 
  umbzdj98 &   1 & supergame2 &  64 &   9 &   3 &   2 &   8 &   6 \\ 
  umbzdj98 &   1 & supergame3 &  48 &  14 &   0 &   0 &   0 &   0 \\ 
  umbzdj98 &   1 & supergame4 & 112 &  17 &   6 &   5 &   6 &   6 \\ 
  umbzdj98 &   1 & supergame5 &  80 &  23 &   3 &   3 &   9 &   9 \\ 
  6ucza025 &   2 & supergame1 &  48 &  25 &   0 &   0 &   0 &   0 \\ 
  6ucza025 &   2 & supergame2 &  64 &  14 &   0 &   0 &   0 &   0 \\ 
  6ucza025 &   2 & supergame3 &  48 &  11 &   0 &   0 &   0 &   0 \\ 
  6ucza025 &   2 & supergame4 & 112 &  16 &   4 &   0 &  12 &   0 \\ 
  6ucza025 &   2 & supergame5 &  80 &   7 &   0 &   0 &   0 &   0 \\ 
  6uv359rf &   2 & supergame1 &  48 &  16 &   1 &   0 &   0 &   0 \\ 
  6uv359rf &   2 & supergame2 &  64 &  17 &   0 &   0 &   0 &   0 \\ 
  6uv359rf &   2 & supergame3 &  48 &  15 &   0 &   0 &   0 &   0 \\ 
  6uv359rf &   2 & supergame4 & 112 &  20 &   0 &   0 &   0 &   0 \\ 
  6uv359rf &   2 & supergame5 &  80 &  21 &   4 &   4 &  10 &  10 \\ 
  irrzlgk2 &   2 & supergame1 &  48 &  12 &   1 &   1 &   3 &   3 \\ 
  irrzlgk2 &   2 & supergame2 &  64 &  17 &   4 &   4 &   4 &   4 \\ 
  irrzlgk2 &   2 & supergame3 &  48 &  12 &   3 &   3 &   5 &   5 \\ 
  irrzlgk2 &   2 & supergame4 & 112 &  19 &  25 &  24 &  42 &  40 \\ 
  irrzlgk2 &   2 & supergame5 &  80 &  26 &  10 &   1 &   3 &   3 \\ 
  j3ki5tli &   2 & supergame1 &  48 &  10 &   0 &   0 &   0 &   0 \\ 
  j3ki5tli &   2 & supergame2 &  64 &  15 &   2 &   0 &   4 &   0 \\ 
  j3ki5tli &   2 & supergame3 &  48 &  13 &   1 &   0 &   3 &   0 \\ 
  j3ki5tli &   2 & supergame4 & 112 &  25 &   2 &   2 &   6 &   6 \\ 
  j3ki5tli &   2 & supergame5 &  80 &  16 &   3 &   0 &   9 &   0 \\ 
  sylq2syi &   2 & supergame1 &  48 &  14 &   2 &   2 &   6 &   6 \\ 
  sylq2syi &   2 & supergame2 &  64 &  16 &   0 &   0 &   0 &   0 \\ 
  sylq2syi &   2 & supergame3 &  48 &  12 &   0 &   0 &   0 &   0 \\ 
  sylq2syi &   2 & supergame4 & 112 &  24 &   4 &   4 &  12 &  12 \\ 
  sylq2syi &   2 & supergame5 &  80 &  19 &   0 &   0 &   0 &   0 \\ 
   \bottomrule
\end{tabular}

	\end{table}

\end{document}
