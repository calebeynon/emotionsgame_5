\documentclass[11pt]{article}
\usepackage{graphicx} % Required for inserting images
\usepackage{multirow}
 
\usepackage[utf8]{inputenc} % Input encoding and font encoding 
\usepackage[margin = 1in]{geometry} % Margins  
\usepackage{setspace} % Setting the spacing between lines
\usepackage{amsthm, amsmath, amsfonts, mathtools, amssymb} % Math packages 
\usepackage{sgame, tikz} % Game theory packages
\usetikzlibrary{trees, calc} % For extensive form games
\usepackage{subfig} % Manipulation and reference of small or sub figures and tables
\usepackage{hyperref} % To create hyperlinks within the document
\hypersetup{
    colorlinks=true, % true if you want colored links
    linktoc=all,     % set to all if you want both sections, subsections, and numbers linked
    linkcolor=blue,  %choose some color if you want links to stand out
}

\def\table{\def\figurename{Table}\figure} % necessary to merge the list of tables with the list of figures
\let\endtable\endfigure % necessary to merge the list of tables with the list of figures

\usepackage{comment}

\renewcommand{\baselinestretch}{1.75}
%\addtolength{\voffset}{-1cm} \addtolength{\hoffset}{-1cm}
%\addtolength{\textwidth}{2.5cm} \addtolength{\textheight}{2cm}

%\title{Infinitely Repeated PD and BOS}
%\author{Paan Jindapon}
%\date{November 2024}

\begin{document}
 
\section*{Data collection}
No, no data have been collected for this study yet.


\section*{Hypothesis}

We plan to conduct a laboratory experiment to study the effects of emotions and information structure on cooperation in infinitely-repeated public goods games. There will be 4 subjects in each group and the group is fixed throughout a supergame. At the end of each stage game, subjects can communicate with other group members via a chat box. We plan to investigate each subject's level of cooperation when they can and cannot monitor each of other group members' contributions to the public good. We will video record each player's facial expression during the experiment.

We expect that (1) each group member communicates more truthfully and cooperates more under perfect monitoring, (2) a subject's level of cooperation in previous periods affects other group members' emotions  detected by facial emotion recognition and text analysis, (3) the detected emotions affect each subject's level of cooperation in subsequent periods.


\section*{Dependent variable}

The dependent variables of interest are (1) the degree of cooperation measured by each subject's contribution to the public good in each period, (2) the scales of emotions detected by facial emotion recognition and text analysis at the end of each period, and (3) whether subjects lie in their chat messages at the end of each period.


\section*{Conditions}
2 conditions. We will adopt a 2-treatment between-subjects design: with and without perfect monitoring. Under perfect monitoring, a subject will be informed about each group member's contribution in the previous period. Without perfect monitoring, a subject will be informed about the total amount of all group members' contributions in the previous period, but not individual contributions. 



\section*{Analyses}

We will use statistic methods to evaluate the treatment effects (with and without perfect monitoring). We will use the iMotions software, specifically Affectiva AFFDEX algorithm, to process the recorded videos in order to detect expressed facial emotions and analyze them. We will use the Natural Language Toolkit (NLTK) available in Python, valence aware dictionary and sentiment reasoner (VADER), and commercially available LLMs % (OpenAI, Anthropic, and Google)  
to detect emotions in chat messages. We will use econometric methods to estimate the effects of other group members' levels of cooperation on subjects' emotions and the effects of detected emotions on levels of cooperation in later periods.

\section*{Outliers and Exclusions}
We will exclude data collected from subjects that do not complete the study.


\section*{Sample Size}
We will collect data from participants at an experimental laboratory at the University of Alabama. For each of the two treatments, we will run 5 sessions with 16 participants in each. Thus, the sample size is expected to be 160.

\section*{Other}
We will ask each participant post-experiment questions about their gender, family background, and majors.

\section*{Name} 
Emotions, information, and cooperation in public goods games

\section*{Type of Project}
Experiment

\end{document}