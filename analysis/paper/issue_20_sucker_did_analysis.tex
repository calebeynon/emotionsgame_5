\documentclass[11pt,oneside]{article}
\usepackage[margin=1in]{geometry}

% Standard Packages
\usepackage{amssymb}
\usepackage{amsmath}
\usepackage{graphicx}
\usepackage{float}
\usepackage[font=footnotesize,labelfont=bf]{caption}
\usepackage{booktabs}
\usepackage{setspace}
\usepackage{placeins}
\usepackage[bookmarks]{hyperref}
\hypersetup{colorlinks = true, allcolors = {black}}

%%%%%%%%% Path resolution (Overleaf: tables/, plots/; Local: ../output/) %%%%%%%%%
\graphicspath{{plots/}{../output/plots/}}
\makeatletter
\def\input@path{{tables/}{../output/tables/}}
\makeatother

\linespread{1.3}

\begin{document}

\section{Dynamic Response to Being Suckered: Event Study Analysis}\footnote{This analysis was generated in part by Claude Opus 4.6 (Anthropic).}

The static sucker regression reveals that being suckered is associated with 6--8 fewer contribution points on average, but this estimate collapses the entire post-event trajectory into a single coefficient. To uncover \textit{when} and \textit{how} the contribution response unfolds, we estimate an event study specification that traces the round-by-round dynamics after a player is suckered.

\subsection{Sample Restriction}

We restrict the analysis to player-segments where the player was suckered in exactly one round, plus never-suckered player-segments as the control group. Under the $< 20$ threshold, this yields 83 single-event treated player-segments and 700 controls (783 total, 3{,}427 player-round observations). Under the $< 5$ threshold, this yields 55 treated and 728 controls. We exclude 17 multi-event player-segments to avoid confounding overlapping treatment effects in the post-period.

Per-round suckering events are re-derived from promise and contribution data, capturing 100 total suckered player-segments (vs.\ 93 from the persistent flag, which misses 7 last-round-only events).

\subsection{Methodology}

Let $\tau_{it}$ denote the number of rounds since player $i$ was suckered. We estimate:
\[
	\text{Contribution}_{it} = \sum_{k \neq 0} \beta_k \cdot \mathbf{1}[\tau_{it} = k] \cdot \text{Suckered}_i + \gamma \cdot \text{Treatment}_i + \alpha_r + \delta_s + \varepsilon_{it}
\]
where $\text{Suckered}_i$ is a time-invariant indicator equal to one for players suckered exactly once in the segment, $\alpha_r$ and $\delta_s$ are round and segment fixed effects, and $\tau = 0$ (the suckering round itself) is the omitted reference period. By definition, suckered players contributed the maximum (25) at $\tau = 0$, making it a natural baseline. Standard errors are clustered at the session--segment--group level. The coefficients $\hat{\beta}_k$ trace the differential contribution path of suckered players relative to controls, with the pre-event coefficients ($k < 0$) serving as a parallel trends check and the post-event coefficients ($k \geq 1$) capturing the response once players learn they were suckered.

\subsection{Contribution Results}

Figure~\ref{fig:did_coefplot} displays the estimated $\hat{\beta}_k$ coefficients. Under the $< 5$ threshold (top panel), pre-event coefficients ($\tau \leq -1$) are close to zero and statistically insignificant, supporting the parallel trends assumption. After the suckering event ($\tau \geq 1$), contributions drop sharply once players learn they were suckered. Under the $< 20$ threshold (bottom panel), the post-event pattern is similar though somewhat attenuated. The full regression estimates are reported in Table~\ref{tab:did_contribution}.

\begin{figure}[H]
	\centering
	\includegraphics[width=0.9\textwidth]{issue_20_did_coefplot}
	\caption{Event study coefficients for the effect of being suckered on contributions. Each point represents $\hat{\beta}_k$ with 95\% confidence intervals. The dashed vertical line separates pre- and post-event periods; $\tau = 0$ (the suckering round) is the omitted reference period.}
	\label{fig:did_coefplot}
\end{figure}

\begin{table}[H]
	\centering
	\caption{Diff-in-Diff: Effect of Being Suckered on Contributions}
	\label{tab:did_contribution}
	{\scriptsize
\begingroup
\centering
\begin{tabular}{lcccc}
   \tabularnewline \midrule \midrule
   Dependent Variable: & \multicolumn{4}{c}{contribution}\\
                                                 & < 20 (Main)    & < 20 (Robust)  & < 5 (Main)     & < 5 (Robust) \\   
   Model:                                        & (1)            & (2)            & (3)            & (4)\\  
   \midrule
   \emph{Variables}\\
   got\_suckered\_20 $\times$ tau\_20 $=$ -6     & -9.418$^{***}$ & -10.05$^{***}$ &                &   \\   
                                                 & (0.9691)       & (0.9606)       &                &   \\   
   got\_suckered\_20 $\times$ tau\_20 $=$ -5     & -3.622         & -4.810         &                &   \\   
                                                 & (3.359)        & (3.170)        &                &   \\   
   got\_suckered\_20 $\times$ tau\_20 $=$ -4     & -4.542         & -5.725$^{*}$   &                &   \\   
                                                 & (3.297)        & (3.367)        &                &   \\   
   got\_suckered\_20 $\times$ tau\_20 $=$ -3     & -0.7982        & -1.997         &                &   \\   
                                                 & (1.845)        & (1.789)        &                &   \\   
   got\_suckered\_20 $\times$ tau\_20 $=$ -2     & -1.627         & -3.050$^{*}$   &                &   \\   
                                                 & (1.771)        & (1.821)        &                &   \\   
   got\_suckered\_20 $\times$ tau\_20 $=$ -1     & -1.428         & -2.948$^{**}$  &                &   \\   
                                                 & (1.288)        & (1.275)        &                &   \\   
   got\_suckered\_20 $\times$ tau\_20 $=$ 1      & -4.958$^{***}$ & -7.144$^{***}$ &                &   \\   
                                                 & (1.427)        & (1.437)        &                &   \\   
   got\_suckered\_20 $\times$ tau\_20 $=$ 2      & -7.132$^{***}$ & -9.228$^{***}$ &                &   \\   
                                                 & (2.383)        & (2.479)        &                &   \\   
   got\_suckered\_20 $\times$ tau\_20 $=$ 3      & -8.069$^{**}$  & -10.36$^{***}$ &                &   \\   
                                                 & (3.550)        & (3.709)        &                &   \\   
   got\_suckered\_20 $\times$ tau\_20 $=$ 4      & -12.88$^{***}$ & -15.83$^{***}$ &                &   \\   
                                                 & (4.234)        & (4.332)        &                &   \\   
   got\_suckered\_20 $\times$ tau\_20 $=$ 5      & -16.33$^{***}$ & -19.43$^{***}$ &                &   \\   
                                                 & (4.006)        & (4.167)        &                &   \\   
   treatment                                     & 1.541$^{***}$  & 1.870$^{***}$  & 1.509$^{***}$  & 1.944$^{***}$\\   
                                                 & (0.5077)       & (0.3085)       & (0.4805)       & (0.2320)\\   
   got\_suckered\_5 $\times$ tau\_5 $=$ -6       &                &                & -9.363$^{***}$ & -10.14$^{***}$\\   
                                                 &                &                & (0.9690)       & (0.9563)\\   
   got\_suckered\_5 $\times$ tau\_5 $=$ -5       &                &                & 0.8891         & -0.7127\\   
                                                 &                &                & (2.111)        & (1.906)\\   
   got\_suckered\_5 $\times$ tau\_5 $=$ -4       &                &                & -1.074         & -2.349\\   
                                                 &                &                & (2.336)        & (2.388)\\   
   got\_suckered\_5 $\times$ tau\_5 $=$ -3       &                &                & 0.2623         & -1.033\\   
                                                 &                &                & (1.620)        & (1.540)\\   
   got\_suckered\_5 $\times$ tau\_5 $=$ -2       &                &                & -2.679         & -4.282$^{**}$\\   
                                                 &                &                & (1.845)        & (1.868)\\   
   got\_suckered\_5 $\times$ tau\_5 $=$ -1       &                &                & -1.335         & -3.043$^{**}$\\   
                                                 &                &                & (1.498)        & (1.483)\\   
   got\_suckered\_5 $\times$ tau\_5 $=$ 1        &                &                & -6.730$^{***}$ & -9.112$^{***}$\\   
                                                 &                &                & (1.878)        & (1.854)\\   
   got\_suckered\_5 $\times$ tau\_5 $=$ 2        &                &                & -12.22$^{***}$ & -14.83$^{***}$\\   
                                                 &                &                & (2.665)        & (2.576)\\   
   got\_suckered\_5 $\times$ tau\_5 $=$ 3        &                &                & -14.57$^{***}$ & -17.40$^{***}$\\   
                                                 &                &                & (3.592)        & (3.561)\\   
   got\_suckered\_5 $\times$ tau\_5 $=$ 4        &                &                & -17.79$^{***}$ & -20.98$^{***}$\\   
                                                 &                &                & (2.295)        & (2.327)\\   
   got\_suckered\_5 $\times$ tau\_5 $=$ 5        &                &                & -16.50$^{***}$ & -20.13$^{***}$\\   
                                                 &                &                & (3.992)        & (3.984)\\   
   \midrule
   \emph{Fixed-effects}\\
   round                                         & Yes            & Yes            & Yes            & Yes\\  
   segment                                       & Yes            & Yes            & Yes            & Yes\\  
   \midrule
   \emph{Fit statistics}\\
   Observations                                  & 3,427          & 2,688          & 3,427          & 2,655\\  
   R$^2$                                         & 0.30136        & 0.48568        & 0.31606        & 0.53705\\  
   \midrule \midrule
   \multicolumn{5}{l}{\emph{Clustered (cluster\_id) standard-errors in parentheses}}\\
   \multicolumn{5}{l}{\emph{Signif. Codes: ***: 0.01, **: 0.05, *: 0.1}}\\
\end{tabular}
\par\endgroup


}
\end{table}

\FloatBarrier

\subsection{Sentiment Results}

We estimate the same event study specification with VADER compound sentiment as the outcome (Table~\ref{tab:did_sentiment} and Figure~\ref{fig:sentiment_trajectory}). The sentiment sample drops round 1 of each supergame (no chat exists), yielding 2{,}230 observations. At $\tau = 1$, sentiment drops by approximately 0.13 ($p < 0.01$) in both threshold specifications---an immediate emotional response to being suckered. The effect partially dissipates in subsequent rounds but remains negative at $\tau = 4$ ($-0.09$, $p < 0.10$). Pre-event sentiment coefficients ($\tau = -2$) are near zero, consistent with parallel trends in the near-event window.

\begin{figure}[H]
	\centering
	\includegraphics[width=0.9\textwidth]{issue_20_sentiment_trajectory}
	\caption{Event study coefficients for the effect of being suckered on chat sentiment (VADER compound score). Each point represents $\hat{\beta}_k$ with 95\% confidence intervals.}
	\label{fig:sentiment_trajectory}
\end{figure}

\begin{table}[H]
	\centering
	\caption{Diff-in-Diff: Effect of Being Suckered on Chat Sentiment}
	\label{tab:did_sentiment}
	{\scriptsize
\begingroup
\centering
\begin{tabular}{lcc}
   \tabularnewline \midrule \midrule
   Dependent Variable: & \multicolumn{2}{c}{sentiment\_compound\_mean}\\
                                                 & < 20 Threshold        & < 5 Threshold \\   
   Model:                                        & (1)                   & (2)\\  
   \midrule
   \emph{Variables}\\
   got\_suckered\_20 $\times$ tau\_20 $=$ -5     & -0.1775$^{**}$        &   \\   
                                                 & (0.0684)              &   \\   
   got\_suckered\_20 $\times$ tau\_20 $=$ -4     & 0.0729                &   \\   
                                                 & (0.0532)              &   \\   
   got\_suckered\_20 $\times$ tau\_20 $=$ -3     & -0.0608               &   \\   
                                                 & (0.0479)              &   \\   
   got\_suckered\_20 $\times$ tau\_20 $=$ -2     & 0.0028                &   \\   
                                                 & (0.0305)              &   \\   
   got\_suckered\_20 $\times$ tau\_20 $=$ -1     & -0.0134               &   \\   
                                                 & (0.0424)              &   \\   
   got\_suckered\_20 $\times$ tau\_20 $=$ 1      & -0.1294$^{***}$       &   \\   
                                                 & (0.0214)              &   \\   
   got\_suckered\_20 $\times$ tau\_20 $=$ 2      & -0.0353               &   \\   
                                                 & (0.0609)              &   \\   
   got\_suckered\_20 $\times$ tau\_20 $=$ 3      & -0.0427               &   \\   
                                                 & (0.0358)              &   \\   
   got\_suckered\_20 $\times$ tau\_20 $=$ 4      & -0.0929$^{*}$         &   \\   
                                                 & (0.0480)              &   \\   
   got\_suckered\_20 $\times$ tau\_20 $=$ 5      & 0.0297                &   \\   
                                                 & (0.0674)              &   \\   
   treatment                                     & $1.19\times 10^{-5}$  & 0.0001\\   
                                                 & (0.0109)              & (0.0109)\\   
   got\_suckered\_5 $\times$ tau\_5 $=$ -5       &                       & -0.1784$^{**}$\\   
                                                 &                       & (0.0686)\\   
   got\_suckered\_5 $\times$ tau\_5 $=$ -4       &                       & 0.0747$^{*}$\\   
                                                 &                       & (0.0442)\\   
   got\_suckered\_5 $\times$ tau\_5 $=$ -3       &                       & -0.0592\\   
                                                 &                       & (0.0567)\\   
   got\_suckered\_5 $\times$ tau\_5 $=$ -2       &                       & -0.0090\\   
                                                 &                       & (0.0295)\\   
   got\_suckered\_5 $\times$ tau\_5 $=$ -1       &                       & 0.0322\\   
                                                 &                       & (0.0393)\\   
   got\_suckered\_5 $\times$ tau\_5 $=$ 1        &                       & -0.1301$^{***}$\\   
                                                 &                       & (0.0223)\\   
   got\_suckered\_5 $\times$ tau\_5 $=$ 2        &                       & -0.0928\\   
                                                 &                       & (0.0763)\\   
   got\_suckered\_5 $\times$ tau\_5 $=$ 3        &                       & -0.0545$^{**}$\\   
                                                 &                       & (0.0252)\\   
   got\_suckered\_5 $\times$ tau\_5 $=$ 4        &                       & -0.0932$^{*}$\\   
                                                 &                       & (0.0483)\\   
   got\_suckered\_5 $\times$ tau\_5 $=$ 5        &                       & 0.0292\\   
                                                 &                       & (0.0671)\\   
   \midrule
   \emph{Fixed-effects}\\
   round                                         & Yes                   & Yes\\  
   segment                                       & Yes                   & Yes\\  
   \midrule
   \emph{Fit statistics}\\
   Observations                                  & 2,230                 & 2,230\\  
   R$^2$                                         & 0.02805               & 0.02590\\  
   \midrule \midrule
   \multicolumn{3}{l}{\emph{Clustered (cluster\_id) standard-errors in parentheses}}\\
   \multicolumn{3}{l}{\emph{Signif. Codes: ***: 0.01, **: 0.05, *: 0.1}}\\
\end{tabular}
\par\endgroup


}
\end{table}

\FloatBarrier

\subsection{Raw Contribution Trajectories}

Figure~\ref{fig:raw_means} complements the regression event study by displaying mean contribution \textit{levels} at each event time $\tau$ for suckered players. The dashed horizontal line marks the control group's grand mean contribution. By construction, suckered players contributed 25 at $\tau = 0$; contributions decline sharply in subsequent rounds. The pre-event trajectory shows suckered players contributing above the control mean before the event, consistent with their high-cooperator profile.

\begin{figure}[H]
	\centering
	\includegraphics[width=0.9\textwidth]{issue_20_raw_means}
	\caption{Mean contributions at each event time $\tau$ for suckered players (line with SE bars). The dashed horizontal line indicates the control group grand mean. At $\tau = 0$, suckered players contributed 25 by definition.}
	\label{fig:raw_means}
\end{figure}

\FloatBarrier

\subsection{Cross-Segment Spillover}

After each supergame, players are reshuffled into new groups, so any contribution change in subsequent supergames reflects a pure behavioral carryover rather than a strategic response to the same groupmates. We estimate two specifications on segments 2--5 (segment 1 has no prior):
\[
	\text{Contribution}_{it} = \beta \cdot \text{SuckeredPrior}_i + \gamma \cdot \text{Treatment}_i + \alpha_r + \delta_s + \varepsilon_{it}
\]
and a decay model replacing the binary indicator with supergame-distance dummies $\mathbf{1}[\text{SegmentsSinceSuckered} = k]$. Table~\ref{tab:spillover} reports the results. The binary spillover coefficient is small and insignificant ($-0.36$, $p > 0.10$ under $< 20$; $-0.87$, $p > 0.10$ under $< 5$), suggesting the within-segment contribution decline does not carry over to new groups after reshuffling.

\begin{table}[H]
	\centering
	\caption{Cross-Segment Spillover: Effect of Prior Suckering on Contributions}
	\label{tab:spillover}
	{\scriptsize
\begingroup
\centering
\begin{tabular}{lcccc}
   \tabularnewline \midrule \midrule
   Dependent Variable: & \multicolumn{4}{c}{contribution}\\
                                          & Binary (< 20) & Decay (< 20) & Binary (< 5) & Decay (< 5) \\   
   Model:                                 & (1)           & (2)          & (3)          & (4)\\  
   \midrule
   \emph{Variables}\\
   suckered\_prior\_segment\_20           & -0.3639       &              &              &   \\   
                                          & (0.5209)      &              &              &   \\   
   treatment                              & 1.357$^{**}$  & 0.8665       & 1.411$^{**}$ & 0.2938\\   
                                          & (0.6506)      & (1.093)      & (0.6374)     & (1.403)\\   
   segments\_since\_suckered\_20 $=$ 1    &               & 2.117$^{*}$  &              &   \\   
                                          &               & (1.225)      &              &   \\   
   segments\_since\_suckered\_20 $=$ 2    &               & 1.752        &              &   \\   
                                          &               & (1.437)      &              &   \\   
   segments\_since\_suckered\_20 $=$ 3    &               & 2.071        &              &   \\   
                                          &               & (1.460)      &              &   \\   
   segments\_since\_suckered\_20 $=$ 4    &               & 1.941        &              &   \\   
                                          &               & (1.696)      &              &   \\   
   suckered\_prior\_segment\_5            &               &              & -0.8726      &   \\   
                                          &               &              & (0.7006)     &   \\   
   segments\_since\_suckered\_5 $=$ 1     &               &              &              & 2.946\\   
                                          &               &              &              & (1.817)\\   
   segments\_since\_suckered\_5 $=$ 2     &               &              &              & 2.906\\   
                                          &               &              &              & (2.048)\\   
   segments\_since\_suckered\_5 $=$ 3     &               &              &              & 3.332\\   
                                          &               &              &              & (2.189)\\   
   segments\_since\_suckered\_5 $=$ 4     &               &              &              & 3.087\\   
                                          &               &              &              & (2.561)\\   
   \midrule
   \emph{Fixed-effects}\\
   round                                  & Yes           & Yes          & Yes          & Yes\\  
   segment                                & Yes           & Yes          & Yes          & Yes\\  
   \midrule
   \emph{Fit statistics}\\
   Observations                           & 2,947         & 1,060        & 2,947        & 705\\  
   R$^2$                                  & 0.27375       & 0.23660      & 0.27475      & 0.26292\\  
   \midrule \midrule
   \multicolumn{5}{l}{\emph{Clustered (cluster\_id) standard-errors in parentheses}}\\
   \multicolumn{5}{l}{\emph{Signif. Codes: ***: 0.01, **: 0.05, *: 0.1}}\\
\end{tabular}
\par\endgroup


}
\end{table}

\FloatBarrier

\subsection{Sucker vs.\ Liar Trajectories}

Figure~\ref{fig:sucker_liar} compares the behavioral trajectories of ``victims'' (suckers) and ``perpetrators'' (liars) around their respective first events. The top row displays mean contributions for suckered players at each event time; the bottom row shows mean contributions for liars---players who promised to cooperate but contributed below the threshold. By construction, suckers contributed 25 at $\tau = 0$ (they cooperated fully while being cheated), while liars contributed below the threshold at $\tau = 0$ (they promised but did not deliver). The two rows reveal opposite sides of the same broken-promise event: suckers reduce contributions afterward (retaliation or distrust), while liars show partial recovery toward the control mean in subsequent rounds.

\begin{figure}[H]
	\centering
	\includegraphics[width=0.95\textwidth]{issue_20_sucker_liar_trajectories}
	\caption{Mean contributions at each event time $\tau$ for suckered players (top) and liars (bottom), faceted by threshold. Dashed horizontal lines indicate the respective control group grand means.}
	\label{fig:sucker_liar}
\end{figure}

\FloatBarrier

\end{document}
