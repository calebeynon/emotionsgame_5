\documentclass[12pt,letterpaper,english]{article}
\usepackage[utf8]{inputenc}
\usepackage[margin=1in]{geometry}
\usepackage{amsmath,amsfonts,amssymb}
\usepackage{graphicx}
\usepackage{booktabs}
\usepackage{caption}
\usepackage{hyperref}
\usepackage{setspace}
\usepackage{lscape}
\usepackage{longtable}
\usepackage{multirow}
\usepackage{float}
\usepackage[authoryear]{natbib}
\usepackage{array}
\usepackage{xcolor}
\usepackage{float}
\usepackage{graphicx}
\usepackage{subcaption}
\usepackage{float}

\doublespacing
\topmargin -0.5 cm
% Set footnote symbols to numbers
\renewcommand{\thefootnote}{\arabic{footnote}}

%\title{Transforming Emotions to Confidence}
\title{Public Goods and Emotions}

\author{
    Caleb Eynon^{1} \and 
  Paan Jindapon^{1} \and 
  Pawonee Khadka^{1} \and
    Laura Razzolini^{1,2}
}


\date

\begin{document}
\begin{spacing}{1}
\maketitle

\footnotetext[1]{Department of Economics, Finance, and Legal Studies, The University of Alabama}
\footnotetext[2]{Center for the Philosophy of Freedom, The University of Arizona}

\section*{Abstract}



\vspace{1em}
\noindent\textbf{Key words} : Experimental Economics, Effect of Emotional factors on Decision Making 

\noindent\textbf{JEL Codes} : C91, D84, D91 
\end{spacing}
\newpage
\section{Introduction}


\section{Experimental Design}
Our study employs a between-subjects design with two treatments that vary the degree of informational visibility in a standard public-goods environment. Across the two treatments, we manipulate whether subjects can observe both the individual contributions of their group members or only the aggregate contribution of the group. 

We conducted a total of 10 sessions, each with 16 participants, yielding 160 subjects overall. Subjects were randomly assigned to a treatment at the session level. Each participant was assigned a unique letter at the beginning of the session. These letters were used to allow participants to observe individual behavior across their group. Subjects were paired into sets of 4 people. In the Information Known treatment, each subject observes the contributions of all three of their group members at the end of every period. In the Information Unknown treatment, subjects see only the total contribution placed into the group account, without any breakdown of who contributed how much. This manipulation allows us to study how transparency—either at the individual or group level—affects cooperative behavior, communication patterns, and contribution dynamics.

The experiment consisted of five segments, each made up of varying number of periods. Groups were re-matched across segments in a way that no subject remained with the same set of partners across segments. 
Each segment followed a random-ending structure based on the continuation probabilities in Lugovskyy et al. (Sequence 1). The sequence of period lengths was (3, 4, 3, 7, 5), totaling 22 periods per segment. Subjects played all 22 periods, resulting in 110 total periods across the five segments. 

Before each decision, subjects participated in a chat phase with the members of their current group. Chat time decreased over the course of a segment. Subjects could freely exchange messages but were prohibited from sharing identifying information or using profanity. All communication occurred through an on-screen chat box.

At the beginning of every period, each subject received 25 points in a private account. Subjects decided how many of these points to allocate to the group account.

After each period, subjects viewed the earnings from that period alongside the group account total (and individual contributions, when applicable). After completing all five segments, one segment was randomly selected to determine each participant’s final payoff.

\section{Results}

\section{Conclusion}

\clearpage
\bibliographystyle{apalike}
\bibliography{reference.bib}

\newpage
\section{Appendix}
\subsection{}

\end{document}

